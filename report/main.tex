%%%%%%%% ICML 2021 EXAMPLE LATEX SUBMISSION FILE %%%%%%%%%%%%%%%%%

\documentclass{article}

% Recommended, but optional, packages for figures and better typesetting:
\usepackage{microtype}
\usepackage{graphicx}
\usepackage{subfigure}
\usepackage{booktabs} % for professional tables

% hyperref makes hyperlinks in the resulting PDF.
% If your build breaks (sometimes temporarily if a hyperlink spans a page)
% please comment out the following usepackage line and replace
% \usepackage{icml2021} with \usepackage[nohyperref]{icml2021} above.
\usepackage{hyperref}

% Attempt to make hyperref and algorithmic work together better:
\newcommand{\theHalgorithm}{\arabic{algorithm}}

% Use the following line for the initial blind version submitted for review:
\usepackage[accepted]{icml2021}
\usepackage{amsmath}
\usepackage{amssymb}
\usepackage{stmaryrd}

% If accepted, instead use the following line for the camera-ready submission:
%\usepackage[accepted]{icml2021}

% The \icmltitle you define below is probably too long as a header.
% Therefore, a short form for the running title is supplied here:
\icmltitlerunning{Reinforcement Learning 2023 Assignment 2}

\begin{document}

\twocolumn[
\icmltitle{Reinforcement Learning 2023, Master CS, Leiden University \\
   Assignment 2 on Deep Q Learning (DQN)}



% It is OKAY to include author information, even for blind
% submissions: the style file will automatically remove it for you
% unless you've provided the [accepted] option to the icml2021
% package.

% List of affiliations: The first argument should be a (short)
% identifier you will use later to specify author affiliations
% Academic affiliations should list Department, University, City, Region, Country
% Industry affiliations should list Company, City, Region, Country

% You can specify symbols, otherwise they are numbered in order.
% Ideally, you should not use this facility. Affiliations will be numbered
% in order of appearance and this is the preferred way.
\icmlsetsymbol{equal}{*}


\begin{icmlauthorlist}
\icmlauthor{Tom Stein (s3780120)}{lu}
\icmlauthor{Andrija Kuzmanov (s3766780)}{lu}
\icmlauthor{Tommaso Ancilli (s3674657)}{lu}
\end{icmlauthorlist}
   
\icmlaffiliation{lu}{Faculty of Science, Leiden University, Leiden, The Netherlands}

\icmlcorrespondingauthor{Tom Stein}{tom.stein@tu-dortmund.de}
\icmlcorrespondingauthor{Andrija Kuzmanov}{andrija.kuzmanov@gmail.com}
\icmlcorrespondingauthor{Tommaso Ancilli}{tommaso.ancilli@student.unisi.it}

% You may provide any keywords that you
% find helpful for describing your paper; these are used to populate
% the "keywords" metadata in the PDF but will not be shown in the document
\icmlkeywords{Reinforcement Learning, Machine Learning}

\vskip 0.3in
]

% this must go after the closing bracket ] following \twocolumn[ ...

% This command actually creates the footnote in the first column
% listing the affiliations and the copyright notice.
% The command takes one argument, which is text to display at the start of the footnote.
% The \icmlEqualContribution command is standard text for equal contribution.
% Remove it (just {}) if you do not need this facility.

%\printAffiliationsAndNotice{}  % leave blank if no need to mention equal contribution
\printAffiliationsAndNotice{\icmlEqualContribution} % otherwise use the standard text.

\begin{abstract}
This assignment report focuses on Deep Q Learning (DQN)
with an application to the CartPole environment. 
The basic concept of DQN is introduced along with the experience replay 
and target network improvements.
The effect of individual hyperparameters is studied empirically 
and through a hyperparameter scan.
A high degree of instability in the training process was observed, 
which is, to some extent, mitigated by specific adjustments of the parameters.
\end{abstract}

\section{Introduction}
\label{sec:introduction}
In this assignment, an agent has to learn how to balance a pole in the vertical position.
The environment presented is a well known and well-defined physical problem, being a reverse pendulum attached to a carriage through a joint.
The goal of this learning task is to keep the pole in place while moving the cart left or right.
The possible action space is made up by a set of only two possible movements ${0,1}$, where $0$: the cart is pushed to the left; $1$: the cart is pushed to the right.
The state space is composed of four values: position and velocity of the cart, angle and angular velocity of the pendulum.
The agent receives a reward of $+1$ for every action performed that results in keeping the pole in an upright state ($\pm 12^\circ$).
The environment is reset to the initial condition every time that the value of the angle between the pole and the vertical line is bigger than $12^\circ$ or when the cart leaves the range horizontal range $(-2.4;2.4)$.
The maximum reward in the environment is limited to 500 because there is no natural end to the balancing problem, and obtaining a reward of 500 demonstrates the ability to balance the pole quite well.

The old paradigm, where all the q-values could be stored in a table and updated directly to determine the optimal policy, is not suitable anymore.
The reason lies in the exponential amount of memory required to store all the possible states (curse of dimensionality).
Given the impossibility of memorizing all the possible states, the agent has to learn how to generalize to unseen data.
This can be done through the application of deep learning.
It is known that an Artificial Neural Network has the property to approximate any function~\cite{Cybenko}, enabling the possibility of inferring an unseen state and having a more compact representation of the solution itself.
The resulting algorithm, generated by the union of Deep learning and Q-learning, is called Deep Q-Network algorithm (DQN).
In this report, the neural network has to approximate the Q-values using a parameterized function.
This approach is also called value-based.

\begin{equation}
    Q_{\theta} :  s \rightarrow q
    \label{eq:value-based-approach}
\end{equation}
where, $s \in S$ where $S$ represents the set of all the possible states and $q$ is the Q-values of the admissible actions.


The remaining chapters are structured as follows.
In \autoref{sec:methodology}, the underlying theories regarding the separate approaches will be shown.
The \autoref{sec:results} is dedicated to the discussion of the optimal configuration for the hyperparameters tuned on the DQN architecture, which exploits the advantages of replay buffer and target network.
In this portion of the report the parameters are tweaked individually, which places this unit at the opposite end of the \autoref{sec:bonus}.
Thereupon, this section is dedicated to study the relationship that each hyperparameter has with the others.
Finally, in \autoref{sec:discussion} the results are summarized and insights into what could have been improved are given.
As a baseline comparison to juxtapose to the different solution will be the random one.
In this one, the agent performs random moves, and the average cumulative reward is set to $22.4$.

\section{Methodology}
\label{sec:methodology}
In this section, we are going to look into the methods used to solve the cart pole balancing problem.
Ideally, we would like to solve the problem with a Q-learning method, since it is easy to implement and to interpret.
However, due to continuous state space it is not possible to store a $Q(s,a)$ value for every state-action pair.
Because of that, we will use deep learning methods to help us estimate the $Q(s,a)$ value for any given state-action pair.
Firstly, we are going to look into the DQN or Deep Q-learning method.
Secondly, we are going to improve the DQN method with experience replay and a target network.

\subsection{DQN}
\label{subsec:dqn}
DQN is one of the most known deep reinforcement learning algorithms.
It is based on the tabular Q-learning method, but instead of obtaining the $Q$ value from a table,
we train (adjust the parameters of) a neural network to predict the values given a certain state.
A neural network is a machine learning structure that contains a large amount of parameters that can be tuned
to represent a certain function.
The DQN agent learns by taking steps in the environment and updating the $Q$ value estimates by using the obtained reward.
The update process is called bootstrapping, as it refines the old estimates using new updates~\cite{DBLP:books/sp/Plaat22}.
In order to update the current $Q(s,a)$ value, we first take a step, we obtain a reward, and we compute the (expected) target value $GT$ of $Q(s,a)$ with the following equation:

\begin{align}
   \label{eq:calculating-target}
    GT = r + \gamma \cdot \max_{a^\prime} Q_{\phi}(s^\prime, a^\prime)
\end{align}

From \autoref{eq:calculating-target} we can see that the target is based on the received reward and the discounted
value of the next state.
The letter $\gamma$ represents the discount factor and the letter $\phi$ represents the parameters of the neural network $Q$.
We can then use the target and our current $Q(s,a)$ estimate to compute the loss using a pre-defined loss function.
A loss function can be any mathematical function that measures the difference between the predicted output and the
actual output.
One of the most common examples when it comes to regression problems is the mean squared error function displayed
in \autoref{eq:mse}, 
where $x$ and $y$ denote the predicted and the actual values, while $n$ denotes the number of values.

\begin{align}
   \label{eq:mse}
   \alpha(x, y) = \frac{1}{n} \cdot \sum_{i = 0}^{n} [(x_i - y_i)^2]
\end{align}
Using the computed loss, we can optimize the parameters $\phi$ in order to minimize the loss.
One of the most common algorithms for optimization is called stochastic gradient descent.
It computes the gradient of the loss function with respect to the parameters $\phi$.
The gradients are then used to push each of the parameters in the opposite direction of the gradient and with that minimize
the loss.
The amount by which each of the parameters is pushed is determined by the learning rate $\alpha$.
The smaller the $\alpha$ the smaller the step taken in the opposite direction of the gradient.

The algorithm~\ref{alg:dqn} displays how a DQN algorithm learns.
Similar to regular Q learning, there are two loops.
In the first loop we define how many episodes we want to execute, and the second loop executes the episode until termination.
While performing an episode, we first need to pick an action based on the $Q$ value estimates, which are obtained by passing the current state $s$ to the neural network.
Because of this, the estimates will be completely random in the beginning, but will improve as we make moves and learn.
We can also use different policies for action-selection, such as epsilon-greedy or the Boltzmann policy
and tune their parameters to incorporate a balance between exploration and exploitation.

After obtaining an action, we can then execute that action inside the environment and obtain the next state $s^\prime$
as well as the reward $r$ for performing that action.
We then compute the $Q(s,a)$ value and the target estimate (using \autoref{eq:calculating-target}).
Our loss is equal to the mean squared difference between the target $Q$ value and the $Q(s,a)$ value we computed.

Once we reach a terminal state, we can calculate the gradient of the loss function with respect to the neural network's parameters.
Those gradients are then provided to the backward pass function that optimizes the parameters.


\begin{algorithm}
   \caption{DQN pseudocode}
   \begin{algorithmic}
      \REQUIRE environment, Qnet, $\alpha \in (0,1]$, $\gamma \in [0,1]$, $\epsilon \in [0,1]$, max\_epoch $\in$ $\mathbf{N}$
      \STATE $s = s_0$ \COMMENT{Initialize start state}
      \FOR{i = 0..max\_epoch}
         \STATE sum\_sq = 0
         \WHILE{$s$ not TERMINAL}
            \STATE $a$ = select\_action(Qnet(s))
            \STATE $s^\prime$, $r$ = env.step($a$)
            \STATE output = max(Qnet($s$))
            \STATE target = $r$ + $\gamma \cdot$ max(Qnet($s^\prime$))
            \STATE sum\_sq += $(target - output)^2$
            \STATE $s = s^\prime$
         \ENDWHILE
         \STATE grad = Qnet.gradient(sum\_sq)
         \STATE Qnet.backward\_pass(grad, $\alpha$)
      \ENDFOR
      \STATE \textbf{Return:} Qnet
   \end{algorithmic}
   \label{alg:dqn}
\end{algorithm}

While the algorithm seems promising, it still faces three important challenges.
Firstly, the convergence of the algorithm depends on the full coverage of the state space, which in the case of the described
problem is just not feasible.
Secondly, there is a strong correlation between subsequent training samples that can lead to poor generalization and overfitting.
Thirdly, we use a bootstrapped target to calculate the loss of each prediction.
The weights of the network that predicts the target change after every optimization step, which can lead to unstable learning~\cite{DBLP:books/sp/Plaat22}.

We can not fully solve the problems posed in the previous paragraph, but we can still try to minimize their impact.
Firstly, the problem of low coverage can be improved by setting a high probability of taking an exploratory step.
The higher the probability, the higher amount of state space the algorithm will explore.
Secondly, we can improve the high correlation problem by implementing a replay buffer, which is described in detail in \autoref{subsec:experience-replay}.
Thirdly, we can implement a secondary neural network that is updated less frequently and only used for target predictions.
This approach is called target network and is explained in \autoref{subsec:target-network}
Finally, the learning can be stabilized by setting a very low learning rate $\alpha$ that will minimize the change between
subsequent target predictions.

\subsection{Experience Replay}
\label{subsec:experience-replay}

\subsection{Target Network}
\label{subsec:target-network}


\section{Results}
\label{sec:results}

\subsection{DQN with ER and TN}
\label{subsec:dqn-with-er-and-tn}

% TODO: Mention catastropic forgetting

\subsection{DQN ablation study}
\label{subsec:dqn-ablation-study}

% Show what happens if we cut out parts of the model architecture.

\section{Discussion}
\label{sec:discussion}
% Discuss results in summary here


% Improvements
%   Use soft update from PolicyNet to Target net instead of hard copy every x epochs
%   Use early stopping to limit effect of overfitting and catastrophic forgetting
%   Use other Optimizer (SGD) and loss metric (Huber-loss)

\section{Hyperparameter Scan - Bonus}
\label{sec:bonus}
In the previous sections, only individual hyperparameters were analyzed in isolation.
This approach does not take into account the dependencies between the different hyperparameters 
such as learning rate and batch size\cite{DBLP:conf/iclr/SmithKYL18}.
For this reason, a comprehensive hyperparameter scan was performed, in which models were trained on random combinations of the hyperparameters. 
In the definition of the hyperparameter space \footnote{\texttt{wandb\_sweep\_config.yaml}} 
no continuous value ranges like $[0.8, 1.0]$ were intentionally used but a preselection of discrete values to be able to group and plot the results more easily.
The space contains about 300 million combinations. 
To search this large space in a reasonable time, several parallel experiments have to be performed. 
For this purpose, the platform Weights and Biases~\cite{wandb} was used to log all training runs and to orchestrate the hyperparameter scan across multiple computers.
The results are shown in \autoref{sec:hyperparameter-scan-results},
% TODO Update link to wandb
but can also be viewed interactively online\footnote{\url{https://api.wandb.ai/links/rl-leiden/xttn9bdo}}.

The results again illustrate the instability caused by wrong parameter selection or catastrophic forgetting, 
since numerous runs only achieve poor results.
% TODO interpret results 
Apart from this, there are also some positive findings, such as .....

% In the unusual situation where you want a paper to appear in the
% references without citing it in the main text, use \nocite
\nocite{DBLP:books/sp/Plaat22}

\bibliography{main}
\bibliographystyle{icml2021}


\appendix
\section{Hyperparameter Scan Results}
\label{sec:hyperparameter-scan-results}

% TODO Update image
\begin{figure*}[ht!]
   \centering
   \includegraphics[width=\textwidth]{assets/hyperparamter-scan/W&B Chart 3_30_2023, 2 24 25 PM.png}
   \caption{Parallel axis plot of the hyperparameter scan. 
      Each line shows a single parameter configuration that was evaluated. 
      The color indicates the performance measured by the average of the last 100 epochs (see legend on the right). 
      Higher average reward (yellow) is better.
   }
   \label{fig_hyperparameter_scan_parallel_axis}
\end{figure*}

\section{Team member contributions}
Andrija implemented the DQN with ER and TN, experiment configs and executed experiments. % TODO
Tom implemented plotting, improved the DQN implementation, executed experiments and did the hyperparameter scan with Weights and Biases.
Tommaso configured the experiments. % TODO
The report was written with equal contribution from everyone.


\end{document}


% This document was modified from the file originally made available by
% Pat Langley and Andrea Danyluk for ICML-2K. This version was created
% by Iain Murray in 2018, and modified by Alexandre Bouchard in
% 2019 and 2021. Previous contributors include Dan Roy, Lise Getoor and Tobias
% Scheffer, which was slightly modified from the 2010 version by
% Thorsten Joachims & Johannes Fuernkranz, slightly modified from the
% 2009 version by Kiri Wagstaff and Sam Roweis's 2008 version, which is
% slightly modified from Prasad Tadepalli's 2007 version which is a
% lightly changed version of the previous year's version by Andrew
% Moore, which was in turn edited from those of Kristian Kersting and
% Codrina Lauth. Alex Smola contributed to the algorithmic style files.
