%%%%%%%% ICML 2021 EXAMPLE LATEX SUBMISSION FILE %%%%%%%%%%%%%%%%%

\documentclass{article}

% Recommended, but optional, packages for figures and better typesetting:
\usepackage{microtype}
\usepackage{graphicx}
\usepackage{subfigure}
\usepackage{booktabs} % for professional tables

% hyperref makes hyperlinks in the resulting PDF.
% If your build breaks (sometimes temporarily if a hyperlink spans a page)
% please comment out the following usepackage line and replace
% \usepackage{icml2021} with \usepackage[nohyperref]{icml2021} above.
\usepackage{hyperref}

% Attempt to make hyperref and algorithmic work together better:
\newcommand{\theHalgorithm}{\arabic{algorithm}}

% Use the following line for the initial blind version submitted for review:
\usepackage[accepted]{icml2021}

% If accepted, instead use the following line for the camera-ready submission:
%\usepackage[accepted]{icml2021}

% The \icmltitle you define below is probably too long as a header.
% Therefore, a short form for the running title is supplied here:
\icmltitlerunning{Reinforcement Learning 2023 Assignment 2}

\begin{document}

\twocolumn[
\icmltitle{Reinforcement Learning 2023, Master CS, Leiden University \\
   Assignment 2 on Deep Q Learning (DQN)}



% It is OKAY to include author information, even for blind
% submissions: the style file will automatically remove it for you
% unless you've provided the [accepted] option to the icml2021
% package.

% List of affiliations: The first argument should be a (short)
% identifier you will use later to specify author affiliations
% Academic affiliations should list Department, University, City, Region, Country
% Industry affiliations should list Company, City, Region, Country

% You can specify symbols, otherwise they are numbered in order.
% Ideally, you should not use this facility. Affiliations will be numbered
% in order of appearance and this is the preferred way.
\icmlsetsymbol{equal}{*}


\begin{icmlauthorlist}
\icmlauthor{Tom Stein (s3780120)}{lu}
\icmlauthor{Tom Stein (s3780120)}{lu}
\icmlauthor{Tom Stein (s3780120)}{lu}
\end{icmlauthorlist}
   
\icmlaffiliation{lu}{Faculty of Science, Leiden University, Leiden, The Netherlands}

\icmlcorrespondingauthor{Tom Stein}{tom.stein@tu-dortmund.de}

% You may provide any keywords that you
% find helpful for describing your paper; these are used to populate
% the "keywords" metadata in the PDF but will not be shown in the document
\icmlkeywords{Reinforcement Learning, Machine Learning}

\vskip 0.3in
]

% this must go after the closing bracket ] following \twocolumn[ ...

% This command actually creates the footnote in the first column
% listing the affiliations and the copyright notice.
% The command takes one argument, which is text to display at the start of the footnote.
% The \icmlEqualContribution command is standard text for equal contribution.
% Remove it (just {}) if you do not need this facility.

%\printAffiliationsAndNotice{}  % leave blank if no need to mention equal contribution
\printAffiliationsAndNotice{\icmlEqualContribution} % otherwise use the standard text.

\begin{abstract}
This assignment report focuses on Deep Q Learning (DQN)
with an application to the CartPole environment. 
The basic concept of DQN is introduced along with the experience replay 
and target network improvements.
A hyperparameter scan is used to empirically compare the performance of
the different models and discuss the results.    
A high degree of instability in the training process was observed, 
which is, to some extent, mitigated by specific adjustments of the parameters.
\end{abstract}

\section{Introduction}
\label{sec_introduction}
In this assignment, an agent has to learn how to balance a pole in the vertical position. The environment \cite{1606.01540} presented is a well known and defined physic problem, being a reverse pendulum. The vertical position is correspond to the unstable equilibrium point and the pole itself is attached through a join to a 
cart. The goal of this learning task is to keep upright the pole while moving the cart left or right.
The possible action space is made up by a set of only two possible movements ${0,1}$, where $0$: the cart is pushed to the left; $1$: the cart is pushed to the right. The state space is composed by four values: position and velocity of the cart, angle and angular velocity of the pendulum.
The agent receives a reward of $+1$ for every action performed resulted in keeping the pole into a equilibrium state. The environment is reset to the initial condition every time that the value of the angle between the pole and the vertical line is bigger than $15°$ or when the cart leaves the range $(-2,4;2,4)$.\\
To tackle this problem tabular methods are not sufficient anymore, since keeping in memory all the possible states is not feasible anymore. Therefore, actions cannot be performed anymore on the basis of the Q-value stored in the table, but we need a way to generalize to unseen states. This can be achieved through the application of an Artificial Neural Network. In this assignment our deep neural network taken as input a state returns the Q values for the actions.\\
As a baseline comparison, we will use a random policy which has a reward around $22$.\\ 
The structure of the following chapters is as follows. In section 2, 3 and 4, for each of the DQN algorithm hyper-parameters optimization will be analyzed, highlighting the optimal configuration. Section 2 covers the basic case of DQN algorithm while, in section 3 the replay buffer is present and in section 4 the target network is considered. After all this architecture has been presented, section 5 is dedicate the comparison and analysis of the different DQN approaches showing their pros and their limitations.


% In the unusual situation where you want a paper to appear in the
% references without citing it in the main text, use \nocite
\nocite{DBLP:books/sp/Plaat22}

\bibliography{main}
\bibliographystyle{icml2021}


\appendix
\section{Hyperparameter Scan Results}


\end{document}


% This document was modified from the file originally made available by
% Pat Langley and Andrea Danyluk for ICML-2K. This version was created
% by Iain Murray in 2018, and modified by Alexandre Bouchard in
% 2019 and 2021. Previous contributors include Dan Roy, Lise Getoor and Tobias
% Scheffer, which was slightly modified from the 2010 version by
% Thorsten Joachims & Johannes Fuernkranz, slightly modified from the
% 2009 version by Kiri Wagstaff and Sam Roweis's 2008 version, which is
% slightly modified from Prasad Tadepalli's 2007 version which is a
% lightly changed version of the previous year's version by Andrew
% Moore, which was in turn edited from those of Kristian Kersting and
% Codrina Lauth. Alex Smola contributed to the algorithmic style files.
